\documentclass[12pt, a4paper]{article}
\usepackage[spanish]{babel}
\usepackage{geometry}
\usepackage[utf8]{inputenc}
\usepackage{lipsum} % Para texto de ejemplo (puedes eliminarlo)
% Remover numeración en secciones
\setcounter{secnumdepth}{-1} % Place this in the preamble

% Set sans-serif font
\usepackage{lmodern}
\renewcommand{\familydefault}{\sfdefault}

% Remover numeración en indice
\usepackage{titlesec}
\usepackage{tocloft}
\renewcommand{\cftsecnumwidth}{0pt}   % Hide section numbers in TOC
\renewcommand{\cftsecaftersnum}{}     % Remove space after section numbers

% Configuración básica de la página
\geometry{a4paper, margin=2.5cm}

% Comando para citar
\newcommand{\miCita}[1]{\cite{#1}}

% Configuración de las referencias
\usepackage[natbibapa]{apacite}
\renewcommand{\bibliographytypesize}{\small}
\AtBeginDocument{\renewcommand{\refname}{Referencias}}

\usepackage{graphicx}
\graphicspath{{/home/gomacodev/Documents/7/Sem_BD/report_4/images/}}

\usepackage{hyperref}

\begin{document}
\begin{titlepage}
\end{titlepage}

% ------------------------------------------------------------------------------
\begin{titlepage}
\begin{center}
 {\Huge\bfseries Report 4\\}
 \vspace{1cm}

 {\huge 12/04/25\\}
 \vspace{2cm}

 {\Large\bfseries Diego Martín Domínguez Hernández}\\[5pt]
 \vspace{2cm}

 {\Large Seminario de bases de datos}\\[5pt]
 \vspace{2cm}

 {\Large Prof. José Antonio Aviña Mendez}\\[5pt]
\end{center}
\end{titlepage}

\section{Introducción}
Tomar en cuenta el caso de estudio en la presentación, crear un diagrama
entidad-relación y un diccionario de datos del mismo.

\newpage
\tableofcontents
\newpage

\section{Diagrama entidad relación}
\includegraphics[width=\textwidth]{er}
\newpage

\section{Diccionario de datos}
\begin{table}[ht]
\centering
\caption{Tabla: USUARIO}
\begin{tabular}{|l|l|p{4.5cm}|p{3.5cm}|}
\hline
\textbf{Atributo} & \textbf{Tipo} & \textbf{Descripción} &
\textbf{Restricciones} \\
\hline
UsuarioID & INT & Identificador único del usuario & Llave primaria \\
\hline
Nombre & VARCHAR(50) & Nombre del usuario & NOT NULL \\
\hline
Apellido & VARCHAR(50) & Apellido del usuario & NOT NULL \\
\hline
RFC & VARCHAR(13) & RFC del usuario para facturación & NOT NULL, UNIQUE \\
\hline
Direccion & VARCHAR(200) & Dirección completa del usuario & NOT NULL \\
\hline
Email & VARCHAR(100) & Correo electrónico del usuario & NOT NULL, UNIQUE \\
\hline
FechaRegistro & DATE & Fecha de registro en la plataforma & NOT NULL \\
\hline
\end{tabular}
\end{table}
\newpage

\begin{table}[ht]
\centering
\caption{Tabla: MEMBRESIA}
\begin{tabular}{|l|l|p{4.5cm}|p{3.5cm}|}
\hline
\textbf{Atributo} & \textbf{Tipo} & \textbf{Descripción} &
\textbf{Restricciones} \\
\hline
MembresiaID & INT & Identificador único de la membresía & Llave primaria \\
\hline
UsuarioID & INT & ID del usuario que contrata la membresía & Llave foránea, NOT
NULL \\
\hline
Tipo & VARCHAR(20) & Tipo de membresía (Standard, Extendida, Premium) & NOT
NULL, CHECK (Tipo IN ('Standard','Extendida','Premium')) \\
\hline
Costo & DECIMAL(10,2) & Costo mensual de la membresía & NOT NULL \\
\hline
FechaInicio & DATE & Fecha de inicio de la membresía & NOT NULL \\
\hline
FechaFin & DATE & Fecha de finalización de la membresía & NOT NULL \\
\hline
\end{tabular}
\end{table}
\newpage

\begin{table}[ht]
\centering
\caption{Tabla: TARJETA}
\begin{tabular}{|l|l|p{4.5cm}|p{3.5cm}|}
\hline
\textbf{Atributo} & \textbf{Tipo} & \textbf{Descripción} &
\textbf{Restricciones} \\
\hline
TarjetaID & INT & Identificador único de la tarjeta & Llave primaria \\
\hline
UsuarioID & INT & ID del usuario dueño de la tarjeta & Llave foránea, NOT NULL
\\
\hline
Numero & VARCHAR(16) & Número de la tarjeta & NOT NULL \\
\hline
Tipo & VARCHAR(20) & Tipo de tarjeta (VISA, MASTERCARD, AMEX) & NOT NULL, CHECK
(Tipo IN ('VISA','MASTERCARD','AMEX')) \\
\hline
FechaExpiracion & DATE & Fecha de expiración de la tarjeta & NOT NULL \\
\hline
\end{tabular}
\end{table}
\newpage

\begin{table}[ht]
\centering
\caption{Tabla: PELICULA}
\begin{tabular}{|l|l|p{4.5cm}|p{3.5cm}|}
\hline
\textbf{Atributo} & \textbf{Tipo} & \textbf{Descripción} &
\textbf{Restricciones} \\
\hline
PeliculaID & INT & Identificador único de la película & Llave primaria \\
\hline
Titulo & VARCHAR(100) & Título de la película & NOT NULL \\
\hline
Duracion & INT & Duración en minutos & NOT NULL \\
\hline
FechaEstreno & DATE & Fecha de estreno de la película & NOT NULL \\
\hline
EsEstreno & BOOLEAN & Indica si la película es estreno y tiene costo adicional &
NOT NULL, DEFAULT FALSE \\
\hline
\end{tabular}
\end{table}
\newpage

\begin{table}[ht]
\centering
\caption{Tabla: SERIE}
\begin{tabular}{|l|l|p{4.5cm}|p{3.5cm}|}
\hline
\textbf{Atributo} & \textbf{Tipo} & \textbf{Descripción} &
\textbf{Restricciones} \\
\hline
SerieID & INT & Identificador único de la serie & Llave primaria \\
\hline
Titulo & VARCHAR(100) & Título de la serie & NOT NULL \\
\hline
FechaEstreno & DATE & Fecha de estreno de la serie & NOT NULL \\
\hline
\end{tabular}
\end{table}
\newpage

\begin{table}[ht]
\centering
\caption{Tabla: EPISODIO}
\begin{tabular}{|l|l|p{4.5cm}|p{3.5cm}|}
\hline
\textbf{Atributo} & \textbf{Tipo} & \textbf{Descripción} &
\textbf{Restricciones} \\
\hline
EpisodioID & INT & Identificador único del episodio & Llave primaria \\
\hline
SerieID & INT & ID de la serie a la que pertenece & Llave foránea, NOT NULL \\
\hline
Titulo & VARCHAR(100) & Título del episodio & NOT NULL \\
\hline
Duracion & INT & Duración en minutos & NOT NULL \\
\hline
NumeroTemporada & INT & Número de temporada & NOT NULL \\
\hline
NumeroEpisodio & INT & Número del episodio en la temporada & NOT NULL \\
\hline
\end{tabular}
\end{table}
\newpage

\begin{table}[ht]
\centering
\caption{Tabla: CATEGORIA}
\begin{tabular}{|l|l|p{4.5cm}|p{3.5cm}|}
\hline
\textbf{Atributo} & \textbf{Tipo} & \textbf{Descripción} &
\textbf{Restricciones} \\
\hline
CategoriaID & INT & Identificador único de la categoría & Llave primaria \\
\hline
Nombre & VARCHAR(50) & Nombre de la categoría (acción, terror, etc.) & NOT NULL,
UNIQUE \\
\hline
\end{tabular}
\end{table}
\newpage

\begin{table}[ht]
\centering
\caption{Tabla: ACTOR}
\begin{tabular}{|l|l|p{4.5cm}|p{3.5cm}|}
\hline
\textbf{Atributo} & \textbf{Tipo} & \textbf{Descripción} &
\textbf{Restricciones} \\
\hline
ActorID & INT & Identificador único del actor & Llave primaria \\
\hline
Nombre & VARCHAR(50) & Nombre del actor & NOT NULL \\
\hline
Apellido & VARCHAR(50) & Apellido del actor & NOT NULL \\
\hline
\end{tabular}
\end{table}
\newpage

\begin{table}[ht]
\centering
\caption{Tabla: DISPOSITIVO}
\begin{tabular}{|l|l|p{4.5cm}|p{3.5cm}|}
\hline
\textbf{Atributo} & \textbf{Tipo} & \textbf{Descripción} &
\textbf{Restricciones} \\
\hline
DispositivoID & INT & Identificador único del dispositivo & Llave primaria \\
\hline
Tipo & VARCHAR(20) & Tipo de dispositivo (smartphone, tablet, laptop, TV) & NOT
NULL, CHECK (Tipo IN ('smartphone','tablet','laptop','TV')) \\
\hline
\end{tabular}
\end{table}
\newpage

\begin{table}[ht]
\centering
\caption{Tabla: CONSUMO}
\begin{tabular}{|l|l|p{4.5cm}|p{3.5cm}|}
\hline
\textbf{Atributo} & \textbf{Tipo} & \textbf{Descripción} &
\textbf{Restricciones} \\
\hline
ConsumoID & INT & Identificador único del consumo & Llave primaria \\
\hline
UsuarioID & INT & ID del usuario que realiza el consumo & Llave foránea, NOT
NULL \\
\hline
ContenidoID & INT & ID del contenido visualizado & NOT NULL \\
\hline
TipoContenido & VARCHAR(10) & Tipo de contenido (Pelicula, Episodio) & NOT NULL,
CHECK (TipoContenido IN ('Pelicula','Episodio')) \\
\hline
FechaHoraInicio & DATETIME & Fecha y hora de inicio del consumo & NOT NULL \\
\hline
FechaHoraFin & DATETIME & Fecha y hora de fin del consumo & NULL \\
\hline
DispositivoID & INT & ID del dispositivo utilizado & Llave foránea, NOT NULL \\
\hline
EsRenta & BOOLEAN & Indica si es una renta de película estreno & NOT NULL,
DEFAULT FALSE \\
\hline
\end{tabular}
\end{table}
\newpage

\begin{table}[ht]
\centering
\caption{Tabla: SESION}
\begin{tabular}{|l|l|p{4.5cm}|p{3.5cm}|}
\hline
\textbf{Atributo} & \textbf{Tipo} & \textbf{Descripción} &
\textbf{Restricciones} \\
\hline
SesionID & INT & Identificador único de la sesión & Llave primaria \\
\hline
UsuarioID & INT & ID del usuario que inicia sesión & Llave foránea, NOT NULL \\
\hline
DispositivoID & INT & ID del dispositivo utilizado & Llave foránea, NOT NULL \\
\hline
FechaHoraInicio & DATETIME & Fecha y hora de inicio de sesión & NOT NULL \\
\hline
FechaHoraFin & DATETIME & Fecha y hora de fin de sesión & NULL \\
\hline
\end{tabular}
\end{table}
\newpage

\begin{table}[ht]
\centering
\caption{Tabla: FACTURA}
\begin{tabular}{|l|l|p{4.5cm}|p{3.5cm}|}
\hline
\textbf{Atributo} & \textbf{Tipo} & \textbf{Descripción} &
\textbf{Restricciones} \\
\hline
FacturaID & INT & Identificador único de la factura & Llave primaria \\
\hline
UsuarioID & INT & ID del usuario al que se factura & Llave foránea, NOT NULL \\
\hline
Fecha & DATE & Fecha de emisión de la factura & NOT NULL \\
\hline
MontoCuotaMensual & DECIMAL(10,2) & Monto por cuota de membresía & NOT NULL \\
\hline
MontoRentasAdicionales & DECIMAL(10,2) & Monto por rentas adicionales de
estrenos & NOT NULL, DEFAULT 0 \\
\hline
MontoTotal & DECIMAL(10,2) & Monto total de la factura & NOT NULL \\
\hline
Estado & VARCHAR(20) & Estado del pago (Al corriente, Deudor) & NOT NULL, CHECK
(Estado IN ('Al corriente','Deudor')) \\
\hline
\end{tabular}
\end{table}
\newpage

% Sección de referencias
\section{Referencias}
\begin{itemize}
    \item Tutorialspoint. (s.f.). \textit{Business Analysis - Requirement
Gathering Techniques}. Recuperado el 13 de abril de 2025, de
\url{https://www.tutorialspoint.com/business_analysis/
business_analysis_requirement_gathering_techniques.htm}

    \item Williams, C. (2018, 11 de septiembre). \textit{Requirements Gathering:
A Step By Step Approach For A Better User Experience (Part 1)}. UsabilityGeek.
\url{https://medium.com/usabilitygeek/requirements-gathering-a-step-by-step-
approach-for-a-better-user-experience-part-1-82be8cad48c2}

    \item Tutorialspoint. (s.f.). \textit{Software Engineering - Software
Requirements}. Recuperado el 13 de abril de 2025, de
\url{https://www.tutorialspoint.com/software_engineering/software_requirements.
htm}
\end{itemize}

\end{document}
