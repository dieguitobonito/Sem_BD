\documentclass[12pt, a4paper]{article}
\usepackage[spanish]{babel}
\usepackage{geometry}
\usepackage[utf8]{inputenc}
\usepackage{lipsum} % Para texto de ejemplo (puedes eliminarlo)
% Remover numeración en secciones
\setcounter{secnumdepth}{-1} % Place this in the preamble

% Set sans-serif font
\usepackage{lmodern}
\renewcommand{\familydefault}{\sfdefault}

% Remover numeración en indice
\usepackage{titlesec}
\usepackage{tocloft}
\renewcommand{\cftsecnumwidth}{0pt}   % Hide section numbers in TOC
\renewcommand{\cftsecaftersnum}{}     % Remove space after section numbers

% Configuración básica de la página
\geometry{a4paper, margin=2.5cm}

% Comando para citar
\newcommand{\miCita}[1]{\cite{#1}}

% Configuración de las referencias
\usepackage[natbibapa]{apacite}
\renewcommand{\bibliographytypesize}{\small}
\AtBeginDocument{\renewcommand{\refname}{Referencias}}

\begin{document}
\begin{titlepage}
\end{titlepage}

% ------------------------------------------------------------------------------
\begin{titlepage}
\begin{center}
 {\Huge\bfseries Report 8: Operaciones SQL\\}
 \vspace{1cm}

 {\huge 18/05/25\\}
 \vspace{2cm}

 {\Large\bfseries Diego Martín Domínguez Hernández}\\[5pt]
 \vspace{2cm}

 {\Large Seminario de bases de datos}\\[5pt]
 \vspace{2cm}

 {\Large Prof. José Antonio Aviña Mendez}\\[5pt]
\end{center}
\end{titlepage}

\newpage
\tableofcontents
\newpage

\section{Introducción}
Explicar qué operaciones se pueden realizar a una base de datos, usando el lenguaje SQL

\section{Inserción}
\subsection{Descripción}
\subsection{Sintaxis}

\section{Actualización}
\subsection{Descripción}
\subsection{Sintaxis}

\section{Borrar}
\subsection{Descripción}
\subsection{Sintaxis}

\section{Consultas}
\subsection{Descripción}
\subsection{Sintaxis}

\section{Crear vistas}
\subsection{Descripción}
\subsection{Sintaxis}

\section{Definir Disparadores (triggers)}
\subsection{Descripción}
\subsection{Sintaxis}


% Sección de referencias
\begin{thebibliography}{9}
\bibitem[Ejemplo, 2024]{ejemplo}
Apellido, N. (2024). \emph{Título del libro}. Editorial.

\bibitem[Teoría, 2023]{teoria}
Otroautor, A. (2023). Artículo importante. \emph{Revista Científica}, 15(2),
123-145.
\end{thebibliography}

\end{document}
