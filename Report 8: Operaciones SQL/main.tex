\documentclass[12pt, a4paper]{article}
\usepackage[spanish]{babel}
\usepackage{geometry}
\usepackage[utf8]{inputenc}
\usepackage{lipsum} % Para texto de ejemplo (puedes eliminarlo)
% Remover numeración en secciones
\setcounter{secnumdepth}{-1} % Place this in the preamble

% Set sans-serif font
\usepackage{lmodern}
\renewcommand{\familydefault}{\sfdefault}

% Remover numeración en indice
\usepackage{titlesec}
\usepackage{tocloft}
\renewcommand{\cftsecnumwidth}{0pt}   % Hide section numbers in TOC
\renewcommand{\cftsecaftersnum}{}     % Remove space after section numbers

% Configuración básica de la página
\geometry{a4paper, margin=2.5cm}

% Comando para citar
\newcommand{\miCita}[1]{\cite{#1}}

% Configuración de las referencias
\usepackage[natbibapa]{apacite}
\renewcommand{\bibliographytypesize}{\small}
\AtBeginDocument{\renewcommand{\refname}{Referencias}}

% Paquetes para escribir código SQL
\usepackage{listings}
\usepackage{xcolor}
\lstset{
  language=SQL,
  basicstyle=\ttfamily\small,
  keywordstyle=\color{blue}\bfseries,
  stringstyle=\color{red!70!black},
  commentstyle=\color{gray}\itshape,
  columns=flexible,
  breaklines=true,
  showstringspaces=false,
  backgroundcolor=\color{gray!10},
  frame=single
}

\begin{document}
\begin{titlepage}
\end{titlepage}

% ------------------------------------------------------------------------------
\begin{titlepage}
\begin{center}
 {\Huge\bfseries Report 8: Operaciones SQL\\}
 \vspace{1cm}

 {\huge 18/05/25\\}
 \vspace{2cm}

 {\Large\bfseries Diego Martín Domínguez Hernández}\\[5pt]
 \vspace{2cm}

 {\Large Seminario de bases de datos}\\[5pt]
 \vspace{2cm}

 {\Large Prof. José Antonio Aviña Mendez}\\[5pt]
\end{center}
\end{titlepage}

\newpage
\tableofcontents
\newpage

\section{Introducción}
Explicar qué operaciones se pueden realizar a una base de datos, usando el lenguaje SQL.

\section{Inserción}
\subsection{Descripción}
Permite agregar nuevos registros a una tabla de la bse de datos creada.
\subsection{Sintaxis}
\begin{lstlisting}
INSERT INTO nombre_tabla (columna1, columna2, ...)
VALUES (valor1, valor2, ...);
\end{lstlisting}
\subsubsection{Ejemplo}
\begin{lstlisting}
INSERT INTO usuarios (nombre, correo, membresia)
VALUES ('Juan Perez', 'juan@mail.com', 'Premium');
\end{lstlisting}

\section{Actualización}
\subsection{Descripción}
Modifica los valores de uno o más campos de registros en una tabla.
\subsection{Sintaxis}
\begin{lstlisting}
UPDATE nombre_tabla
SET columna1 = valor1, columna2 = valor2
WHERE condicion;
\end{lstlisting}
\subsubsection{Ejemplo}
\begin{lstlisting}
UPDATE usuarios
SET membresia = 'Standard'
WHERE nombre = 'Juan Perez';
\end{lstlisting}

\section{Borrar}
\subsection{Descripción}
Elimina uno o varios registros de una tabla según la condición dada.
\subsection{Sintaxis}
\begin{lstlisting}
DELETE FROM nombre_tabla
WHERE condicion;
\end{lstlisting}
\subsubsection{Ejemplo}
\begin{lstlisting}
DELETE FROM usuarios
WHERE nombre = 'Juan Perez';
\end{lstlisting}

\section{Consultas}
\subsection{Descripción}
Permite consultar y recuperar información almacenada en una o más tablas.
\subsection{Sintaxis}
\begin{lstlisting}
SELECT columna1, columna2
FROM nombre_tabla
WHERE condicion;
\end{lstlisting}
\subsubsection{Ejemplo}
\begin{lstlisting}
SELECT nombre, membresia
FROM usuarios
WHERE membresia = 'Premium';
\end{lstlisting}

\section{Crear vistas}
\subsection{Descripción}
Una vista es una tabla virtual basada en el resultado de una consulta \textit{SELECT}.
\subsection{Sintaxis}
\begin{lstlisting}
CREATE VIEW nombre_vista AS
SELECT columna1, columna2
FROM nombre_tabla
WHERE condicion;
\end{lstlisting}
\subsubsection{Ejemplo}
\begin{lstlisting}
CREATE VIEW vista_usuarios_premium AS
SELECT nombre, correo
FROM usuarios
WHERE membresia = 'Premium';
\end{lstlisting}

\section{Definir Disparadores (triggers)}
\subsection{Descripción}
Es un conjunto de instrucciones que se ejecutan automáticamente en respuesta a ciertos eventos (\textit{INSERT, UPDATE, DELETE})
\subsection{Sintaxis}
\begin{lstlisting}
CREATE TRIGGER nombre_trigger
AFTER INSERT ON nombre_tabla
FOR EACH ROW
BEGIN
   -- instrucciones
END;
\end{lstlisting}
\subsubsection{Ejemplo}
\begin{lstlisting}
CREATE TRIGGER registrar_insercion_usuario
AFTER INSERT ON usuarios
FOR EACH ROW
BEGIN
   INSERT INTO log_usuarios (usuario_id, fecha)
   VALUES (NEW.id, NOW());
END;
\end{lstlisting}

\
% Sección de referencias
\section{Referencias}
\begin{itemize}
    \item W3Schools. \emph{SQL Tutorial}. Recuperado de \url{https://www.w3schools.com/sql/}
\end{itemize}
\end{document}
